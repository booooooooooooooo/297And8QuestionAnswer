%%%%%%%%%%%%%%%%%%%%%%%%%%%%%%%%%%%%%%%%%%%%%%%%%%%%%%%%%%%%%%%%
%%%%%%%%%%%  Example usage of sjsuthesis.cls %%%%%%%%%%%%%%%%%%%
%%%%%%%%%%%%%%%%%%%%%%%%%%%%%%%%%%%%%%%%%%%%%%%%%%%%%%%%%%%%%%%%

\documentclass[modernstyle,12pt]{sjsuthesis}


%%%%%%%%%%%%%%%%%%%%%%%%%%%%%%%%%%%%%%%%%%%%%%%%%%%%%%%%%%%%%%%%
%%%%%%%%%%%    load any packages which are needed    %%%%%%%%%%%
%%%%%%%%%%%%%%%%%%%%%%%%%%%%%%%%%%%%%%%%%%%%%%%%%%%%%%%%%%%%%%%%

% these are typical

\usepackage{latexsym}		% to get LASY symbols
\usepackage{epsfig}		% to insert PostScript figures
\usepackage{graphicx}           % to insert any other kind of figure

% these are for math stuff

\usepackage{amsmath}	% AMS math features (e.g., eqn alignment)
\usepackage{amssymb}	% Various weird symbols
\usepackage{amsfonts}	% Various useful math fonts
\usepackage{amsthm}	% Fancy theorem-type environments

% Convention: everything (except pictures) is numbered inside a single
% sequence, starting again in each section.  This makes things much
% easier to read.

\newtheorem{thm}{Theorem}[section]
\newtheorem{lem}[thm]{Lemma}
\newtheorem{cor}[thm]{Corollary}
\newtheorem{conj}[thm]{Conjecture}
\newtheorem*{main}{Main Theorem}
\theoremstyle{definition}
\newtheorem{defn}[thm]{Definition}
\newtheorem{rem}[thm]{Remark}
\newtheorem{exmp}[thm]{Example}
\newtheorem{ques}[thm]{Question}


%%%%%%%%%%%%%%%%%%%%%%%%%%%%%%%%%%%%%%%%%%%%%%%%%%%%%%%%%%%%%%%%
%%%%%%%%%%%%       all the preamble material:       %%%%%%%%%%%%
%%%%%%%%%%%%%%%%%%%%%%%%%%%%%%%%%%%%%%%%%%%%%%%%%%%%%%%%%%%%%%%%

\title{A Question Answering System on SQuAD Dataset Using an End-to-end Neural Network}

\author{BO}{LI}

\degree{Master of Science}		%  #1 {long descr.}
	{M.S., Computer Science}		%  #2 {short descr.}

\degreemonth{May}
\degreeyear{2018}

\dept{Department of}			%  #1 {designation}
     {Computer Science}		        %  #2 {name}

\advisor{Dr.}				%  #1 {title}
	{Chris Pollett}			%  #2 {name}
\advisorOrg{Department of Computer Science}

\reader{Dr.~Suneuy Kim}		        %  2nd person to sign thesis
\readerOrg{Department of Computer Science}

\readerThree{Dr.~David Taylor}		%  3rd person to sign thesis
\readerThreeOrg{Department of Computer Science}

% you can optionally add \readerFour and \readerFive as well

%\readerFour{Dr.~Who Dat}		%  4th person to sign thesis
%\readerFourOrg{Department of Physics, Harvard Univ.}

% NOTE: to get the front matter single spaced, put \singlespacing
% at the start of the abstract text

\abstract{TODO}



% acknowledgements page is optional

\acknowledgements{

Thanks to
}

% the following options can be enabled or disabled

%\ToCisShort	% a 1-page Table of Contents

% Default: List of figures will be printed
% Uncomment the \emptyLoF line to skip the list of figures
%\LoFisShort	% a 1-page List of Figures
%\emptyLoF	% no List of Figures at all

%\LoTisShort	% a 1-page List of Tables
% \emptyLoT	% no List of Tables at all


%%%%%%%%%%%%%%%%%%%%%%%%%%%%%%%%%%%%%%%%%%%%%%%%%%%%%%%%%%%%%%%%%
%%%%%%%%%%%%%%%       BEGIN DOCUMENT...         %%%%%%%%%%%%%%%%%
%%%%%%%%%%%%%%%%%%%%%%%%%%%%%%%%%%%%%%%%%%%%%%%%%%%%%%%%%%%%%%%%%

% the \begin{document} will generate all the prologue material
% (signature page, TOC, etc.); if you want to control this
% behavior, uncomment one of the following lines:
%
% \SuspendPrologue    % disables the prologue entirely
% \SimplePrologue     % prints only title, abstract, TOC, TOF


% the following command will cause a draft version and the
% current date to be printed in the header area
%
% \draftVersion{1}


\begin{document}

\raggedright          % as per SJSU GS&R guidelines June 2010
\parindent=30pt       % restores indentation

% \singlespacing      % uncomment to print single spaced (e.g., for drafts)


% document body goes here

\chapter{Introduction}
\chapter{Background}
\section{Word Feature Vector}
Word Feature Vector (WFC) was firstly came up with by CITEBengio2003. A word feature vector carries information about the likeliness of each word from the vocabulary to be its neighbour. A word feature represents a word according to its relationship with all words in the vocabulary.

In CITEBengio2003, the word feature vectors of a given text is learned from training a neural probabilistic language model(NPLM) on the text. The NPLM is illustrated in CITEfigurenlpmfromcs297report, where V is the vocabulary, $w_t$ represents a word from $V$, and the matrix $C$ contains the word feature vectors of all words in $V$. Each instance of the training set is a sequence of words $w_1,...,w_T$. The purpose of NPLM is to train a model $f$ such that $ f(w_t, ..., w_{t-n+1}) = \hat{P}(w_t | w_{t-1},...,w_{t-n+1})$. The computation of $f(w_t, ..., w_{t-n+1})$ is divided into two parts.
First, we map each $w$ to a WFC by selecting the corresponding row in $C$ to get $x=(C(w_{t-1}),... ,C(w_{t-n+1}))$. Second, we get $f(w_t, ..., w_{t-n+1})$ through $y=b+W\cdot x + U\cdot tanh(d + H\cdot x)$ and $ f(w_t, ..., w_{t-n+1}) = \frac{e^{y_{w_t}}}{\sum_{i}^{}e^{y_i}}$. The loss function to minimize is $L = -\frac{1}{T}\sum _{t}^{} \log{f(w_t, ..., w_{t-n+1})}$.


WFC enables learning dependencies on longer sentences. In comparison, n-grams only considers around 3 words in practice due to the computation infeasibility when n is larger. As such, a natural language model built on WFC can consider more context information than that built on n-grams. More context generally brings more more accurate result.

The usage of WFC is far beyond simply predicting a word's neighbours. In practice, WFC is a common way to represent the words in an end-to-end neural network model which does a natural language processing task.
\section{Recurrent Neural Networks (RNNs)}
Recurrent neural networks (RNNs) CITE(Rumelhart et al., 1986a) are neural networks with sequence-based specialization. They are used for modeling sequential data.

TODO:figure10.2fromdeeplearningbook

In this research, I use two types of design patterns. The first type is a RNN with recurrent connections between hidden states and the sequence of hidden states are the output of the RNN. The second type is also a RNN with recurrent connections between hidden states, but the last hidden state is the output of the RNN.

I use Long Short Term Memory (LSTM) cell and Gated Recurrent Unit (GRU) cell instead of the basic RNN cell. They prevent the RNNs from vanishing after a few rollings, as illustrated in CITEAPPENDIX.
\section{Bidirectional RNNs}

The RNNs of Section CITEREFERPREVIOUSSECTION, rolls from left to right. As such, the $h_t$ only contains context information from $x_1$ to $x_t$, but does not contain context information from $x_{t+1}$ to the end. However, in most sequence-to-sequence tasks, we want $h_t$ to contain the information of the whole sequence. Bidirectional RNNs makes this possible. In a bidirectional RNN, one RNN rolls from left to right, another RNN rolls from right to left. At time t, using both $h1_t$ and $h2_t$ would get context information from the whole sequence.

TODO:FITUREOFBIDIREDTIONALRNNmyresearchuses

\section{Encoder-Decoder Sequence-to-Sequence Architecture}

Sequence-to-sequence means the input to the model is a sequence, and the output from the model is also a sequence. An encoder-decoder architecture can be applied to do this task. The process of understanding the input sequence is considered as encoding the input sequence to some vectors. The process of generating output is considered as decoding the vectors.

TODO:figureOfEncoderDecoderSequenceToSequenceArch

The answer questioning task is a sequence-to-sequence task. The input sequences are questions and passages, and the output sequences are answers.

\section{Attention Mechanism}

Attention mechanism was first came up with by CiteBahdanau et al. (2015) in the application of neural machine translation. Neural machine translation is also a sequence-to-sequence task. The input sequence is some words in one language, and the output sequence is the same content in another language. In the encoder-decoder sequence-to-sequence architecture for neural machine translation, a RNN encodes the input sequence to one state, and the decoder decodes the state to the output sequence. However, a big problem here is the state cannot contain all the information of a long input sequence. The attention mechanism was invented to let the decoding process knows about the input sequence. CiteDeepLearningbookChapter12 summarizes the attention mechanism used in neural machine translation into three components:
\begin{itemize}
\item{A process that reads raw data.}
\item{A list of feature vectors storing the output of the reader.}
\item{A process that exploits the content of the memory to sequentially perform a task, at each time step having the ability put attention on the content of one memory element (or a few, with a different weight).}
\end{itemize}

TODO:fitureOfAtentionInNMT

Although the attention mechanism was invented for doing machine learning task, it can be applied to many other tasks of natural language processing. I will explain how attention mechanism is used in this project in Chpter CITEDESIGNCHAPTER




\section{Pointer Network}

CiteOriolVinyalsetal discovered Pointer Network in 2015. Pointer Network enables decoder to output a sequence of indexes in input sequence. It uses attention weight as a probability distribution to select a member from the input sequence as the output token. This usage is different from that in neural machine translation where the attention weight is used to get a weighted average vector from input sequence and use it as context information to generate the output.

TODO:figureOfPointerNetwork




\chapter{Design}

Considering the limited computation resource, I choose CiteWang'spaper as the main reference of my design. The model in CiteWang'spaper reflects a general architecture to do question answering task, and the number of layers is not too many to train. In Section citeBaselineArchitecture I introduce CiteWang'spaper and how I modify it to get my baseline architecture. Then I describe my three changes on the baseline architecture to analyze how the baseline architecture works.

\section{Baseline Architecture}

Wang and Jiang CiteWang'spaper proposed an encoder-decoder sequence-to-sequence architecture for the question answering task on SQuAD dataset. Each instance of training data includes one passage, one question and one answer. The passage is a sequence of tokens, the question is a sequence of tokens, and the answer is two indices indicating the start and end position in passage. Recall that each answer is part of the corresponding passage in SQuAD dataset.

Before feeding training data into model, tokens in passages and questions are vectorized to word feature vectors. As such, some pre-trained word feature vector matrix is an additional dataset in need.

The vectorized training data is feed into the encoder. The encoder includes two layers - preprocessing layer and bi-directional match-LSTM layer. In preprossing layer, a LSTM network runs over each passage word feature vector sequence and outputs a sequence of LSTM hidden states. The same LSTM is used to encode each question word vector sequence to a sequence of LSTM hidden states.

$$H^p = \overrightarrow{LSTM}(P)$$
$$H^q = \overrightarrow{LSTM}(Q)$$

where

 $$P\in R^{d \times p}: passage$$
 $$Q\in R^{d \times q}: question$$
 $$H^p\in R^{l \times p}: encoded\ passage$$
 $$H^q\in R^{l \times q}: encoded\ question$$
 $$p: length \ of\ passage$$
 $$q: length\ of\ question$$
 $$l: dimension\ of\ LSTM\ hidden\ states$$
 $$d: dimension\ of\ word\ feature\ vector$$

In bi-directional match-LSTM layer, a LSTM equipped with passage-question attention, which is called match-LSTM, is used to encode each sequence passage hidden states and the pairing sequence of question hidden states together to a sequence of hidden states of the match-LSTM. To be specific,

$$\overrightarrow{G} = tanh(W^qH^q + (W^p{h_i}^p + W^r\overrightarrow{{h_{i-1}}^r} + b^p) \otimes e_q)$$
$$\overrightarrow{\alpha _i} = softmax(w^t\overrightarrow{G_i} + b \otimes e_q)$$


where

$$W^q, W^p, W^r\in R^{l \times l} $$
$$b_p, w\in R^{l}  $$
$$b \in R $$
$$\overrightarrow{{h_{i-1}}^r}\in R^{l}: one\ column\ of\ H_p  $$

and

\[ \overrightarrow{z_i} =
\begin{bmatrix}
{h_i}^p \\
H^q\overrightarrow{ {\alpha _i}}^T \\
\end{bmatrix}
\in R^{2l}
\]
$$\overrightarrow{{h_i}^r} = \overrightarrow{LSTM}(\overrightarrow{z_i}, \overrightarrow{{h_{i-1}}^r}).$$

After iterating between getting attention vector $\overrightarrow{\alpha _i}$ and getting hidden state ${{h_{i}}^r}$ $p$ times, we get $[{{h_{1}}^r}, ..., {{h_{p}}^r}]$. Concatenate them to get

$$\overrightarrow{H_r} = [{{h_{1}}^r}, ..., {{h_{p}}^r}] \in R^{l \times p}.$$

Then go over $H_p$ from right to left to get $\overleftarrow{H_r}$. Concatenate $\overrightarrow{H_r}$ and $\overleftarrow{H_r}$ to get the final output of encoding process

\[ H_r =
\begin{bmatrix}
\overrightarrow{H_r} \\
\overleftarrow{H_r} \\
\end{bmatrix}
\in R^{2l \times p}.
\]

The decoding process includes only one layer - Answer Pointer layer. This layer is motivated by the Pointer Net in paper citevinyals2015pointer I have discussed in Section CITE. Wang and Jiang CITE proposed two ways to design this layer. Here I only explain the boundary model.

$$F_k = tahn(VH_r + (W^a{h_{k-1}}^a +  b^a) \otimes e_p)$$
$$\overrightarrow{\beta _k} = softmax(v^tF_k + c \otimes e_p)$$


where
$$V \in R^{l \times 2l}$$
$$W^a\in R^{l \times l} $$
$$b_a, v\in R^{l}  $$
$$c \in R $$
$$\overrightarrow{{h_{k-1}}^a}\in R^{l}: hidden\ state\ at\ positiom\ i\ of\ answer\ LSTM  $$

and answer LSTM is


$$\overrightarrow{{h_k}^a} = \overrightarrow{LSTM}(H^r\beta _k^T, h_{k-1}^a)$$

By iterating between the attention mechanism and the answer LSTM two times, we could get $\beta _0$ and $\beta _1$. Let $a_s$ denote the start index of the answer, and $a_e$ denote the end index, then we have

$$p(a|H^r) = p(a_s|H_r)p(a_r|H_r)=\beta _{0, a_s} \times \beta_{1, a_e}$$

where $$\beta_{k, j} = jth\ token\ of\ \beta _k$$

To train the model, the loss function

$$J(\theta) = -\frac{1}{N}\sum_{i=1}^{N} \log{p(a^n|H^r)} $$

is minimized.

\section{First Change to Baseline Architecture}

\section{Second Change to Baseline Architecture}

\section{Third Change to Baseline Architecture}

\chapter{Implementation}

\section{Batch Gradient Descent}

To achieve batched training, paragraphs should be padded to a same length. Similarly, questions are also padded the a same length. As such, the model in implementation has some difference with the theoretical one explained in \ref{theoreticalModel}.

In preprocessing layer,
$$H^p = H^p \circ passage\_mask$$
$$H^q = H^q \circ question\_mask$$

In match-LSTM layer,
$$\overrightarrow{\alpha _i} = softmax( (w^t\overrightarrow{G_i} + b \otimes e_q) \circ question\_mask)$$


$$H_r = H_r \circ passage\_mask$$

In Ans-Ptr layer,
$$\overrightarrow{\beta _k} = softmax( (v^tF_k + c \otimes e_p) \circ passage\_mask)$$


\section{Tensorflow Graph}
Describe the graph of baseline architecture.

TODO:conceptionalFigureOfTfGraphOfBaseline

How the graph is shared among train, valid and test


\section{Implementation Architecture}

TODO:figureOfWholePipeline

\chapter{Experiments}
\section{Data}
I use Stanford Question Answering Dataset (SQuAD) to do experiemnts.  It contains 100,000+ question-answer pairs on 500+ articles. The training set and dev set are visible to users. However, the test set is hidden. To perform my experiments, I split the training set into my training and dev set, and use the dev set as my test set. After such splitting, my training set contains 78839 question-answer pairs, my dev set contains 8760 question-answer pairs, and my test set contains 10570 question-answer pairs.

 Regarding to data processing, I use a Python natural language processing library {\tt nltk} to tokenize raw strings in json file into passage-question-answer triplets in the format of word token sequences.

Then I face two ways to represent the word tokens. The first way is to query the corresponding word vector of each token from the GloVe embedding matrix and turn the word sequences into vector sequences.  The second way is to make a vocabulary and turn each word sequence to an index sequence based on the vocabulary. At the same time, a smaller embedding matrix is made from the original GloVe embedding matrix. The index of each token in this matrix is same with that in vocabulary. The index sequences and the smaller embedding matrix is fed into the neural network. Since the index sequences require much less memory than the vector sequences, the second way saves a lot of memory than the first way when large data set is used.

The next step is to pad passage and question, as mentioned in Chapter TODO:CITE. There is a {\tt pad} token in vocabulary, and the corresponding word vector is zero vector.

At last, split data into batches.

After training, the embedding matrix becomes part of the tensorflow graph, and is unchanged during testing. Also, the vocabulary remains unchanged during testing. Words not found in vocabulary during testing are treated as unknown word. The word vector of unknown words is the average of all vectors of known words in my training and dev set.

TODO: table: size of train, dev and test set

TODO: table: size of json file, voc file, embedding matrix file, token\_id file of train and valid data

TODO: figure: work flow of data processing

\section{Settings}
At the first step, I refer to the experimental settings of TODOCiteInnerPeacegithub to set some of my parameters. The dimension of GloVe word vectors is 100, the size of hidden units is 64, the regularization scale of L2-regularization is 0.001 and the batch size is 32. For passage length and question length, I plot out their distributions and find out that 400 is a reasonable cut point for passage, and 30 is a reasonable cut point for question.

TODO: figure of passage length, question length

I use the default settings of tensorflow for adam optimizer. I sample out 200 instances from train set to estimate train error, and from dev set to estimate validation error. The normalization boundary to clip gradients is set as 5.

Then I do several experiments on different learning rate, since this is the most important parameters. I initially planed to do this on the match architecture, but due to some code bugs at that time, I actually do the experiments on match\_change1 architecture.I started the experiments from 0.002, which is used by TODOCiteInnerPeacegithub. It performs well. Then I try several larger value. However, all the larger values I try do not perform well. So I set learning rate to 0.002.

TODO: figure of different lr


The settings that work well on one architectures also work well on other three architectures. This really sames a lot of time and money.

I used Tesla K80 12 GB Memory 61 GB RAM 100 GB SSD GPU to train the four models.

TODO: table of all experimental settings

I use F1 score and exact match score to evaluate the performance of each architecture. TODO:explain F1 and em


\section{Results}

For match architecture, the model converges after around 8 epochs. Match\_change1 and match\_change2 perform similar to match. For match\_change2, the model converges after around 4 epochs.  including training time, loss change, score change etc. For all four architectures, training each epoch costs around 2 hours on the Tesla GPU I use.

TODO:figureoftrainingstat

Figure TODO:citefigure shows the testing results of on test data set of the four architectures. In the original dev set, which is our test set, each question corresponds to multiple answers. This makes sense since in reality we might have several different ways to answer a same question. We use two ways to calculate the scores. The first way is choose the first answer, the second way is calculating scores for each answer and choose the best score.In either way, match, match\_change1 and match\_change2 perform similarly, but match\_change3 behaves much worse than the other three.


TODO:figureoftestingresult

Interactive Testing on match architecture shower some interesting reuslts.

TODO:figures of interactive testing result.

\section{Analysis}
We can draw several interesting insights from the results. First, more context information increases, or at least does not decrease the accuracy. Second, the ways to add context information are various. Third, duplicate context might not be necessary.





\chapter{Conclusion}

\chapter{Getting started with \LaTeX\ and our thesis template}

Template revised June 13, 2011.

\section{Why \LaTeX?}

There are several reasons you should write your thesis using \LaTeX\
and our math department style file.

\begin{itemize}
\item Our style file has been customized, with the help of Graduate
  Studies and Research, to fulfill the SJSU thesis formatting
  requirements without any extra effort.
\item Once you get the hang of using \LaTeX, typesetting and
  formatting mathematics is very natural.  (In comparison, typesetting
  mathematics in most word-processing systems can be painful and
  time-consuming.)
\item In \LaTeX, numbering and referencing of chapters, sections,
  equations, and so on, is automatic, allowing for great flexibility
  in making changes.  For example, if you want to insert an extra
  section or an extra chapter in the middle of your thesis, or reorder
  the sections within a chapter, in \LaTeX\ this takes just a few
  minutes!  (See Section \ref{sect:latex-math} for an example of how
  this works.)
\item The same \LaTeX\ code you use to typeset the math in your thesis
  can be moved painlessly to a slide format for a poster presentation
  or for your thesis defense.
\item Using \LaTeX\ lets you concentrate on content and forget about
  appearance of your text.
\item Finally, \LaTeX\ is often used in technical/engineering
  workplaces to write manuals, so learning \LaTeX\ is potentially a
  valuable job skill.
\end{itemize}

The downsides of using \LaTeX\ are that you will have to put in some
time learning a new system and a slightly different way of thinking
about word processing/typesetting; and to some extent, you will have
to trade off ease of use (i.e., the point-and-click, drag-and-drop
approach) for precise control.  You may also occasionally have to
experiment and fiddle with things to make them work.  Nevertheless,
for many people, this investment of time is well worth it, and often
ends up saving time in the end.



\section{Typesetting math in \LaTeX}
\label{sect:latex-math}

When you write math in \LaTeX, it is often naturally divided into {\it
  environments\/} that correspond to the way mathematics is organized.
For example, our math department thesis format has environments
corresponding to standard definitions, theorems, and proofs.

\begin{defn}
Definitions look one way.
\end{defn}

\begin{thm}\label{thm:sample}
On the other hand, theorems look like this.
\end{thm}

\begin{proof}
And their proofs like this.  Check out how equations look:
\begin{equation}\label{eq:ftc}
\int_a^b \dfrac{df}{dx} \,dx = f(b)-f(a)
\end{equation}
(for $f$ such that $f'(x)=\frac{df}{dx}$ is continuous on $[a,b]$, for
example).
\end{proof}

Note that the equation number appearing above is not typed in by hand,
but actually generated by the \LaTeX\ program itself.  The great thing
about automatic numbering is that you can refer to Theorem
\ref{thm:sample} (in the original \LaTeX\ code, something like
%
\verb@Theorem \ref{thm:sample}@%
%
), and it will always refer to the correct theorem, even if you change
the position of the theorem or rearrange the sections.  (Try it on the
{\tt example.tex\/} file!)  Something similar is true for equations,
though referring to them works a little differently; e.g., see
\eqref{eq:ftc} (in \LaTeX, something like \verb@\eqref{eq:ftc}@).



\section{Getting \LaTeX}

One thing about how \LaTeX\ works that may be different for you is
that, unlike a word processing program, the file that you type into is
different from the file that you print out.  Specifically, for most
up-to-date \LaTeX\ systems, you type into a file called something like
{\tt thesis.tex}, using a type of program called a {\it text editor},
you run the \LaTeX\ program to produce a PDF file called something
like {\tt thesis.pdf}, and you print out {\tt thesis.pdf}.  Some
systems keep the text editor separate from the \LaTeX\ part, and some
systems integrate them, but either way, you should be aware that the
parts are there somewhere.

Here are some popular ways to run \LaTeX, listed by type of computer,
with prices as of June 2007.

\begin{itemize}
\item {\bf PC, separate text editor and \LaTeX.}  Download the
  following programs:
  \begin{enumerate}
  \item Get \LaTeX\ for {\bf free\/} at {\tt http://www.miktex.org}.
    {\it One tricky point:\/} When you install MiKTeX, at some point,
    you will be asked what your favorite paper is; make sure you
    answer ``Letter paper'' and not the British paper size ``A4''
    (which is the default!).
  \item Get the WinEdt text editor at {\tt http://www.winedt.com}.
    The creators of WinEdt request a {\bf \$30\/} shareware fee, which
    we urge you to pay to help keep WinEdt going.
  \end{enumerate}

\item {\bf PC, integrated environment.}  Buy PCTeX at {\tt
    http://www.pctex.com}.  This is {\bf \$150\/} (student price) for
  the ``Publisher'' version, which is recommended if you have the
  money, and {\bf \$50\/} (student price) for the ``Writer'' version,
  which is recommended (sort of, see below) if you don't.

\item {\bf Mac, integrated environment.} Download the TeXShop system
  for {\bf free\/} at {\tt
    http://www.uoregon.edu/$\sim$koch/texshop/}.

\item {\bf Linux or Mac OS X, separate text editor and \LaTeX.}  If
  you run Linux at home, \LaTeX\ is almost certainly already installed
  on your machine, as is the text editor {\tt emacs}.  (Of course, if
  you're a Linux user, you're the sort of person who doesn't need
  instructions.)

  Also, if you have a Mac running OS X, and you're fond of the command
  line, you can download \LaTeX\ and {\tt emacs\/} for {\bf free\/}
  from the usual sources --- but if your taste in computing is {\it
    that\/} perverse, you really don't need instructions on how to do
  that.  (Spoken from experience: This style guide was produced using
  command-line \LaTeX\ and {\tt emacs}.)
\end{itemize}

More specifically, if you own a PC, it seems that the best option for
most people is to use WinEdt and MikTeX.  If you are strongly
computer-phobic, try the ``Publisher'' version of PCTeX, which can
produce PDF output, whereas the ``Writer'' version cannot.  Only if
you are both strongly computer-phobic and short on cash should you buy
the ``Writer'' version of PCTeX.



\section{First step: Compile the example}

The next step is to run \LaTeX\ on the file {\tt example.tex\/} in
this directory, to see if you can reproduce this file, {\tt
  example.pdf}.  (Make a copy of this version in a safe place first.)
We refer you to the instructions that go along with your software for
details, but the most important distinction is between systems that
use PDF-style latex (``pdflatex'' systems) and systems that use plain
(old-school) latex.  This distinction becomes important if you want to
put pictures in your thesis, so please read ahead a bit before you do
this.  It is also important to note that if you want to use JPEG
pictures (e.g., digital photos) directly, you should use a pdflatex
system like WinEdt/MikTeX or TeXShop.

\begin{itemize}
\item WinEdt/MikTeX (a pdflatex system): Open {\tt example.tex\/}
  using WinEdt, and press the ``bear'' (pdflatex) button on the menu
  bar.

\item PCTeX (a plain latex system): Open {\tt example.tex\/} using
  PCTeX, modify the {\tt figure\/} environments in the section
  ``Including pictures in your thesis'', below, and press the
  ``latex'' button (see PCTeX instructions).

\item TeXShop (a pdflatex system): Open {\tt example.tex\/} using
  TeXShop, and press the ``Typeset'' button on the menu bar.

\item Command-line systems (Linux, Mac OS X, etc.): You can choose
  either the command {\tt pdflatex example\/} or the command {\tt
    latex example}.  If you choose the latter, the output file will be
  a file called {\tt example.dvi}, which you can convert to PDF
  (printable) format by the commands:
\begin{verbatim}
dvips example -o
ps2pdf example.ps
\end{verbatim}
  You may want to put that in a macro.  Note that pdflatex is more
  direct, but you may prefer to use plain latex to include certain
  kinds of pictures; see Section \ref{sect:pictures}.
\end{itemize}



\section{References and bibliography}

The way \LaTeX\ handles references is that bibliographic information
goes in a file like the file {\tt refs.bib\/} included in this
directory.  Then, when you refer to a book in the course of your
thesis, you type something like \verb@Artin~\cite{Artin:Algebra}@,
press some appropriate sequence of typesetting buttons (see below),
and hey presto, the reference Artin~\cite{Artin:Algebra} appears
appropriately formatted.  Again, you just concentrate on content, and
\LaTeX\ and our style file take care of making things look good!

Though to be precise, that is not completely true.  In your text, you
{\it do\/} need to refer to papers and books in a standard style,
which goes something like:
\begin{quotation}
  (name(s) of author(s)) [reference tag]
\end{quotation}
The basic idea is that you refer to a reference by the name(s) of its
author(s), and the [name] tag is supplied by \LaTeX.  A good rule of
thumb is that if the tag is erased, your text should still make sense.
For example: ``In Pomerance~\cite{Pomerance:Sieve}, we see
that\ldots'' or ``For our standard algebra reference, we use
Artin~\cite[Ch.\ 2]{Artin:Algebra}.''  It may be a little strange, at
first, to refer to a book by its author's name, but that is the
standard practice.  (See the bibliography section at the end of this
document to see what the tags are referring to.)

Your references are stored in a file called {\tt refs.bib}, which you
need to edit separately.  Precise details of the format of a .bib file
can be found elsewhere, but for the purposes of your thesis, you will
almost certainly be able just to imitate one of the entries in the
version of {\tt refs.bib\/} included in this package.  These include
the following types of bibliographic references.  (Compare the source
code of {\tt example.tex\/} with how these references appear in print
in the text of this document and the bibliography section.)
\begin{itemize}
\item References to articles in a mathematical or scientific journal
  look like Pomerance~\cite{Pomerance:Sieve} or Bektemirov, Mazur,
  Stein, and Watkins~\cite{BMSW:EllipticCurves}.
\item You can refer to a book, like Artin~\cite{Artin:Algebra}, or a
  chapter within that book, as in Artin~\cite[Ch.\ 4]{Artin:Algebra}.
  Articles and books are the most common types of references to appear
  in theses.
\item You may need to refer to an unpublished paper or preprint, like
  Fermat~\cite{F:LastTheorem}.
\item The entry Asimov~\cite{Asimov:Again} illustrates two points.
  First, this kind of .bib entry is used for articles that appear in
  non-journal collections of articles, like conference proceedings.
  The other point is that \LaTeX\ will often turn capital letters in
  titles to lowercase letters (this is part of many standard
  bibliographic styles).  Therefore, if something like a proper noun
  (e.g., someone's name) appears in a title, you need to put it inside
  brackets in your .bib file to tell \LaTeX\ not to lowercase it.
\item Finally, you may need to cite a web page or other informally
  published electronic material.  As always, the main point is that
  you need to give enough information for the reader to check your
  citation, so make sure you include a very complete URL for all web
  pages.  See the entry for Dunn~\cite{Dunn:Computability}.
\end{itemize}

Now, the first time you latex a file that contains bibliographic
citations, the citations may look like [??].  To make the references
in a document appear properly, try the following sequence of commands,
or the analogous button-pushing sequence.

\begin{verbatim}
latex example
bibtex example
latex example
latex example
\end{verbatim}

WinEdt, PCTeX, and TeXShop all do this sequence automatically when
necessary, but it never hurts to bibtex and latex a few more times
just to make sure, especially if, for example, you're just about to
turn in the final draft of your thesis.



\section{Making pictures for your thesis}

Things get a little more complicated if you want to use pictures in
your thesis.  Before you start, you need to answer three questions.

\begin{enumerate}
\item Do you want to use precise, line-drawing, diagram-type pictures;
  do you want to use more fuzzy, photograph-like pictures; or do you
  need both types?  If you need to make the first kind of picture,
  you'll want to get a {\it vector graphics editor}, and if you want
  to make the second kind of picture, you'll want to get a {\it raster
    (bitmap) graphics editor}.

\item Will you be using picture formats compatible with pdflatex,
  namely, PDF and JPEG/.jpg files, or will you be using the picture
  format compatible with plain latex, namely, EPS?  Note that you will
  have to choose one approach or the other and stick with it; there is
  no mixing the two systems.

\item Will you be using some kind of well-known mathematical software
  (e.g., MATLAB or Maple) to make pictures?  If so, which of the above
  output formats does your program produce?
\end{enumerate}

Once you've answered those questions, you can decide which program(s)
to buy or download.

\begin{itemize}
\item If you need a vector graphics editor (diagram-type pictures),
  the standard commercial program is Adobe Illustrator ({\bf \$200\/}
  student price, PC and Mac, Spartan Bookstore).  Xfig is a good {\bf
    free\/} program that is close enough to Illustrator for most
  purposes; it can be downloaded for Linux or Mac OS X at {\tt
    http://xfig.org/}.  (For Mac OS X, you may first have to install
  the X Window system from the Optional Packages folder of the Mac OS
  X Install DVD.)  Winfig, which is basically the Windows/PC version
  of xfig, is available at {\tt
    http://www.schmidt-web-berlin.de/winfig/} for a {\bf \$25\/}
  shareware registration fee.

\item If you need a raster/bitmap graphics editor (photograph-type
  pictures), the standard commercial program is Adobe Photoshop ({\bf
    \$300\/} student price, PC and Mac, Spartan Bookstore).  There is
  an excellent {\bf free\/} program named, unfortunately, GIMP (GNU
  Image Manipulation Program) that is downloadable at {\tt
    http://www.gimp.org/}, in PC and Mac/Linux versions.  (Again, for
  Mac OS X, you will first need to install the X Window system.)
\end{itemize}



\section{Including pictures in your thesis}
\label{sect:pictures}

Again, the precise method of including pictures depends on whether
your version of \LaTeX\ is a modern pdflatex system (e.g.,
WinEdt/MikTeX, TeXShop) or an old-school plain latex system (e.g.,
PCTeX).  In any case, the basic idea is that each picture goes in a
{\tt figure\/} environment, whose placement is somewhat independent of
the regular placement of other text in your thesis.

With pdflatex-style systems, you can include either PDF (vector
graphics) pictures or JPEG (raster/bitmap graphics) pictures.  Figure
\ref{f:smileypdf} shows the result of including a PDF file in a
\LaTeX\ document, and Figure \ref{f:hijpg} shows the result of
including a JPEG file in a \LaTeX\ document.  Note that figures will
often {\it not\/} appear in the location in your text where you
originally ask them to appear, so your text should not depend on a
figure appearing in one particular location.  In other words, your
main text should still make sense if the figure is removed completely.
(As a rule, however, there should be a reference to a figure soon
before it appears in your thesis, much as we have referred to Figures
\ref{f:smileypdf} and \ref{f:hijpg} here just before they actually
appear.)

\begin{figure}[htbp]\centering
  % put a comment character (%) at the beginning of the next line for
  % plain latex systems like PCTeX
  \includegraphics{smiley.pdf}
  \caption{A PDF figure}
  \label{f:smileypdf}
\end{figure}

\begin{figure}[htbp]\centering
  % put a comment character (%) at the beginning of the next line for
  % plain latex systems like PCTeX
  \includegraphics{hi.jpg}
  \caption{A JPEG figure}
  \label{f:hijpg}
\end{figure}

Finally, as promised, Figure \ref{f:smileyeps} shows the result of
including an EPS file in a \LaTeX\ document.  Again, note that this
will not work for pdflatex-style systems, so make sure you plan ahead.

\begin{figure}[htbp]\centering
  (uncomment this next line for plain latex systems)
  %\includegraphics{smiley.eps}
  \caption{An EPS (encapsulated Postscript) figure}
  \label{f:smileyeps}
\end{figure}

Note that currently, this file automatically produces a List of
Figures in its front matter (things before Chapter 1).  If fewer than
three figures appear in your thesis (e.g., if you do not use any
figures), you should remove the List of Figures by changing the line
at the beginning of this file that says
\verb@%\emptyLoF@ to \verb@\emptyLoF@.



\section{Tables}
\label{sect:tables}

Tables are handled much the same as figures, though since tables are
easier to typeset, they can be made within \LaTeX\ without the use of
some external program.  Table \ref{tab:NSA} gives an example of a
table and how it is labeled by its caption.  Note that as a rule,
table captions are placed above the table and not below, though this
can vary with different standard styles.  Again, tables should be
referred to in the text soon before they actually appear.

\begin{table}[htbp]\centering
  \caption{Starting salaries at NSA}
  \label{tab:NSA}
  \begin{tabular}{|r|l|} \hline
    Highest degree earned & Average starting salary \\ \hline\hline
    Bachelor's & \$42,207 \\
    Master's & \$58,203 \\
    Ph.\ D. & \$86,172 \\ \hline
  \end{tabular}
\end{table}

If you have more than 3 tables within your thesis, you will need to
have a List of Tables, similar to the List of Figures mentioned in
Section \ref{sect:pictures}.  Just put a {\tt \%} in front of the
\verb@\emptyLoT@ command at the beginning of this file to make the
List of Tables appear.



\section{Abstract and introduction}

When you have completed the technical content of your thesis, one of
the very last things you should do is to write your abstract and your
introduction.  Both are summaries of your thesis, but roughly
speaking, the abstract is for experts, and the introduction is for
novices.  More precisely:

\begin{itemize}
\item Your abstract should give a very brief (at most 150 words, by
  Graduate Studies and Research rules) technical summary of what you
  did in your thesis.  You do not need to give much, if any, context
  for what you did; the point is to let an expert know very quickly
  what is contained in your thesis.
\item Your introduction should also summarize what is in your thesis,
  but should provide context and background for the reader unfamiliar
  with your subject.  A good introduction answers questions like: What
  is the point of all this?  What previous work has been done on this
  topic?  What is new or original here, and what is previously
  well-known?
\end{itemize}

If you make your entire Chapter 1 your introduction (recommended),
here is a good way to structure it.  (Of course, this is only one way
to write an introduction, and you are not required to use this
approach, but it does seem to work quite well.)
\begin{itemize}
\item Section 1.1: Try to state the main problem you are considering
  to the extent that you can without going too much into details of
  definitions.  Give context and history for the problem: What has
  been done previously, who did it, and when?  Try to give a clear
  idea of what is new here, what is standard material, and what is
  somewhere in the middle (new exposition of well-known facts).
\item Section 1.2: Give an overview of what is contained in your
  thesis.  A seemingly boring but actually very useful way to do this
  is to expand on your table of contents: ``In Chapter 2, we\ldots.
  In Chapter 3, we\ldots.''
\item Section 1.3: Define any unusual notation you might be using.
\end{itemize}
Of course, one of the great things about doing your thesis in \LaTeX\
is that you can insert a new Chapter 1 near the end of your writing
process, and the rest of your thesis will automatically renumber
itself to match!



\section{Writing mathematics well}
\label{sect:well}

What we have discussed so far should give you some idea of how to
produce a reasonable-looking document using our \LaTeX\ document
class.  What we have not really discussed, however, is how to write
mathematics {\it well}.  This is a complicated and subtle issue that,
to be honest, many mathematicians have not mastered.  We therefore
mention only a few key points.

\begin{enumerate}
\item Theses in mathematics should be written using the usual
  conventions of formal writing.  For example, do not use
  contractions.
\item The strangest conventions of written mathematics are the use of
  the ``mathematical we'' (first person plural) and the present tense.
  The basic idea is that, in a situation where a scientific paper
  might use the passive voice and the past tense (``The groups of
  order 144 were studied''), a math paper uses ``we'' in the present
  tense (``We study the groups of order 144'').  In general, all of
  your thesis except the introduction should be written in the present
  tense.
\item In an introduction, the past tense is useful for distinguishing
  between work previously done on your topic and what you are doing in
  your thesis: ``Wiles and Taylor showed that\ldots.  We show
  that\ldots.''
\item The biggest key is to be organized.  Give your work structure
  and put it into some kind of outline framework, instead of just
  telling a story 5-year-old-style: ``And then\ldots.  And
  then\ldots.''
\item A few small things:
  \begin{itemize}
  \item Words like Theorem x.y, Lemma x.y, Chapter x, Section x.y,
    should all be capitalized.
  \item Do not start a sentence with a mathematical symbol.  For
    example, instead of ``$G$ is therefore abelian,'' use ``Therefore,
    $G$ is abelian'' or ``The group $G$ is therefore abelian.''
  \item There are different ways to deal with equations mod $n$,
    depending on whether the equation appears in text or in a
    stand-alone equation.  In text, an equation might look like
    $2+2\equiv 21$ (mod~$n$), and as a stand-alone equation, the same
    equation would become
    \begin{equation}
      \label{eq:modn}
      2+2\equiv 21 \pmod{n}.
    \end{equation}
    Compare the source code for {\tt example.tex\/} to see how those
    two equations are produced.
  \end{itemize}
\end{enumerate}

For more ideas on the style and substance of writing mathematics, see
{\it The Chicago Manual of Style\/}~\cite[Ch.\ 14]{Chicago} (note how
this citation shows how \LaTeX\ handles references with no author or
editor) and Knuth, Larrabee, and Roberts~\cite{KLR:WritingMath}.  A
free version of the latter source can be found in the file {\tt
  mathwriting.pdf\/}; pages 1--6 give a good list of tips to help you
get started.


\section{Last words}

You may have noticed that all of the chapter and section titles in
this document are capitalized in what is known as {\it sentence
  style\/}; for example, the title of Section \ref{sect:well} is
``Writing mathematics well'', not ``Writing Mathematics Well,'' as one
would do in {\it headline style}.  You can use whichever style you
prefer, as long as you are consistent, though if you have no
preference, it seems to be easier to use sentence-style
capitalization.

For more about \LaTeX, try the following references.  All can be
purchased online at the usual places or at many technical bookstores.

\begin{itemize}
\item Griffiths and Higham~\cite{GH:Latex} is a very good beginner's
  guide, and is especially notable for its brevity (just 94 pages).

\item Lamport~\cite{L:Latex} is the classic guide, and while it is
  neither entirely for beginners nor exhaustive as a reference, it
  strikes a good balance between the two extremes.

\item Kopka and Daly~\cite{KD:Latex} is encyclopedic and thorough.
  Not at all meant for beginners, but a good reference.

\item For the truly hardcore who want to try things like rewriting
  style files, there is always Knuth~\cite{K:TeX}.
\end{itemize}

Please send corrections and suggestions to Tim Hsu
(hsu@math.sjsu.edu), who maintains this template.  And have fun!






\chapter{Book the first: Recalled to life}

This other chapter is just to illustrate how the table of contents,
etc., work.  Note that chapters after the first chapter all get page
numbers.  (Though one of the best parts of using our \LaTeX\ style
file is that you no longer have to think about things like that!)

\section{A section of words}

It was the best of times, it was the worst of times, it was the age of
wisdom, it was the age of foolishness, it was the epoch of belief, it
was the epoch of incredulity, it was the season of Light, it was the
season of Darkness, it was the spring of hope, it was the winter of
despair, we had everything before us, we had nothing before us, we
were all going direct to Heaven, we were all going direct the other
way- in short, the period was so far like the present period, that
some of its noisiest authorities insisted on its being received, for
good or for evil, in the superlative degree of comparison only.

\section{Just another section}

To show how the table of contents stuff works, and how multiple
sections work.


% if you want to keep macros or chapters in separate files
% you can do that and include them with \input like this:

%\input macros.tex
%\input ch1.tex
%\input ch2.tex


%%%%%%%%%%%%%%%%%%%%%%%%%%%%%%%%%%%%%%%%%%%%%%%%%%%%%%%%%%%%%%%%%%%
%%%%%%%%%%%%%%%%%%%%%%%  Bibliography %%%%%%%%%%%%%%%%%%%%%%%%%%%%%
%%%%%%%%%%%%%%%%%%%%%%%%%%%%%%%%%%%%%%%%%%%%%%%%%%%%%%%%%%%%%%%%%%%

\bibliographystyle{amsalpha}	% or "siam", or "alpha", or "abbrv"
				% see other styles in
				% texmf/bibtex/bst

%\nocite{*}		% uncomment to list all refs in database,
			% cited or not.

\bibliography{refs}		% assumes bib database in "refs.bib"

%%%%%%%%%%%%%%%%%%%%%%%%%%%%%%%%%%%%%%%%%%%%%%%%%%%%%%%%%%%%%%%%%%%
%%%%%%%%%%%%%%%%%%%%%%%%  Appendices %%%%%%%%%%%%%%%%%%%%%%%%%%%%%%
%%%%%%%%%%%%%%%%%%%%%%%%%%%%%%%%%%%%%%%%%%%%%%%%%%%%%%%%%%%%%%%%%%%

\appendix	% don't forget this line if you have appendices!

\chapter{Gratuitous Appendix}
Nothing to see here.

%\input appA.tex


%%%%%%%%%%%%%%%%%%%%%%%%%%%%%%%%%%%%%%%%%%%%%%%%%%%%%%%%%%%%%%%%%%%
%%%%%%%%%%%%%%%%%%%%%%%%   THE END   %%%%%%%%%%%%%%%%%%%%%%%%%%%%%%
%%%%%%%%%%%%%%%%%%%%%%%%%%%%%%%%%%%%%%%%%%%%%%%%%%%%%%%%%%%%%%%%%%%

\end{document}
